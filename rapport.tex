\documentclass[12pt]{article}
\usepackage[utf8]{inputenc}
\usepackage[french]{babel}
\usepackage{amsmath, amssymb}
\usepackage{graphicx}
\usepackage{lmodern}
\usepackage{geometry}
\geometry{margin=2.5cm}

\title{Concepts mathématiques – Estimation de probabilité de COVID-19 à partir de symptômes}
\author{ANDRIATSIFERANA No Kanto Lorida}
\date{}

\begin{document}

\maketitle

\section*{1. Modélisation du problème}

On souhaite estimer la probabilité qu’un patient soit positif au COVID-19, étant donné les symptômes qu’il présente.

C’est un problème de classification binaire :
\begin{itemize}
    \item Classe 1 : \textbf{COVID Positif} ($y = 1$)
    \item Classe 0 : \textbf{COVID Négatif} ($y = 0$)
\end{itemize}

Chaque symptôme est une variable binaire (présent ou absent), représentée dans un vecteur $\mathbf{x} = (x_1, x_2, ..., x_n)$.

\section*{2. Hypothèse : modèle de Naive Bayes}

Nous utilisons le théorème de Bayes :
\[
P(y=1 \mid \mathbf{x}) = \frac{P(\mathbf{x} \mid y=1) \cdot P(y=1)}{P(\mathbf{x})}
\]

Puisque $P(\mathbf{x})$ est commun à toutes les classes, on peut comparer uniquement les numérateurs :
\[
P(y=1 \mid \mathbf{x}) \propto P(y=1) \cdot \prod_{i=1}^n P(x_i \mid y=1)
\]
\[
P(y=0 \mid \mathbf{x}) \propto P(y=0) \cdot \prod_{i=1}^n P(x_i \mid y=0)
\]

\section*{3. Estimation des probabilités conditionnelles}

À partir du jeu de données :
\begin{itemize}
    \item $P(y=1)$ : proportion de cas positifs
    \item $P(x_i=1 \mid y=1)$ : fréquence du symptôme $i$ chez les cas positifs
    \item $P(x_i=0 \mid y=1) = 1 - P(x_i=1 \mid y=1)$
\end{itemize}

Pour éviter les divisions par zéro, on applique un lissage de Laplace :
\[
P(x_i = 1 \mid y) = \frac{n_i + 1}{n + 2}
\]

\section*{4. Calcul pratique : log-vraisemblance}

Pour éviter les sous-flux numériques, on calcule en logarithmes :
\[
\log P(y=1 \mid \mathbf{x}) = \log P(y=1) + \sum_{i=1}^n \log P(x_i \mid y=1)
\]
\[
\log P(y=0 \mid \mathbf{x}) = \log P(y=0) + \sum_{i=1}^n \log P(x_i \mid y=0)
\]

Ensuite, on applique la normalisation log-softmax :
\[
P(y=1 \mid \mathbf{x}) = \frac{e^{\log P_1}}{e^{\log P_1} + e^{\log P_0}}
\]

\section*{5. Interprétation des résultats}

\begin{itemize}
    \item Plus un utilisateur coche de symptômes fréquents chez les cas positifs, plus la probabilité estimée augmente.
    \item Le score final est une probabilité entre 0 \% et 100 \%, facile à interpréter.
    \item On applique un seuil simple (ex. 50\%) pour alerter ou rassurer l’utilisateur.
\end{itemize}

\section*{6. Exemple concret}

Symptômes cochés : Toux sèche, Fièvre, Perte de goût.

\begin{itemize}
    \item $P(\text{Toux sèche} \mid y=1) = 0{,}7$
    \item $P(\text{Fièvre} \mid y=1) = 0{,}8$
    \item $P(\text{Perte de goût} \mid y=1) = 0{,}75$
\end{itemize}

Calcul :
\[
\log P(y=1) + \log(0{,}7) + \log(0{,}8) + \log(0{,}75)
\]
Et de même pour $y=0$, puis normalisation.

\end{document}
